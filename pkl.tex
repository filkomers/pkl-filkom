%%%%%%%%%%%%%%%%%%%%%%%%%%%%%%%%%%%%%%%%%%%%%%%%%%%%%%%%%%%%%%%%%%%%%%%
% Latex file abstraction to pkl class
%%%%%%%%%%%%%%%%%%%%%%%%%%%%%%%%%%%%%%%%%%%%%%%%%%%%%%%%%%%%%%%%%%%%%%%
\documentclass{pkl}

% ------------------------------------------------------------
% Add your own needed package here
% ------------------------------------------------------------
\usepackage[titles]{tocloft} % add prefix for images and tables
\renewcommand\cftfigpresnum{Gambar\  }
\renewcommand\cfttabpresnum{Tabel\   }

\usepackage{hyperref} % add hyperlink to the document
\newlength{\mylenf}
\settowidth{\mylenf}{\cftfigpresnum}
\setlength{\cftfignumwidth}{\dimexpr\mylenf+2em}
\setlength{\cfttabnumwidth}{\dimexpr\mylenf+2em}

\usepackage[labelfont=bf]{caption} % Bold Face for image description
\usepackage{subfig} % images side by side

\usepackage{float} % force placing figure here
\usepackage{csquotes}
\usepackage{enumitem} % paragraph inside itemize

\renewcommand*{\arraystretch}{1.8} % allow space in math formula

% -----------------------------------------------------------------
% longtable
% -----------------------------------------------------------------

\usepackage{longtable} % add table
% itemiize inside longtable
% \usepackage{array,booktabs,enumitem}% http://ctan.org/pkg/{array,booktabs,enumitem}
% \newcolumntype{P}[1]{>{\endgraf\vspace*{-\baselineskip}}p{#1}}

% avoid overfull in longtable
% TODO overfull still occured
\usepackage{array}
\newcolumntype{P}[1]{>{\raggedright\let\newline\\\arraybackslash\hspace{0pt}}p{#1}}

% -----------------------------------------------------------------
% Custom source code style
% -----------------------------------------------------------------
\usepackage[newfloat,chapter]{minted}

% allow caption and label in minted
% https://tex.stackexchange.com/questions/254044/caption-and-label-on-minted-code
\newenvironment{code}{\captionsetup{type=listing}}{}
\SetupFloatingEnvironment{listing}{name=Tabel Kode}

\usepackage{tcolorbox}
\tcbuselibrary{listings,minted,skins,breakable}
\newtcblisting{ignasicblock}[1][]{%
  breakable,
  colback=white,
  colframe=black,
  colbacktitle=white,
  sharp corners,
  enhanced,
  listing engine=minted,
  listing only,
  left=10mm,
  title=Source Code,
  halign title=center,
  overlay={\draw[line width=.5mm] ([xshift=8mm]frame.south west)
    -- ([xshift=8mm]frame.north west);
    \node[right] at (title.west) {No};},
  minted style=colorful,
  minted language=Python,
  minted options={%
    linenos=true,
    fontsize=\footnotesize,
    numbersep=6mm,
    texcl=true,
    breaklines=true,
    autogobble=true},
  coltitle=black,
  #1
}

\definecolor{codebg}{rgb}{0.95,0.95,0.95}
\renewcommand\theFancyVerbLine{\footnotesize\arabic{FancyVerbLine}}

% -----------------------------------------------------------------
% Font style
% -----------------------------------------------------------------
\usepackage{fontspec} % use custom font

\setmainfont[Path=/home/yourusername/.local/share/fonts/font-windows/calibri/,
BoldItalicFont=calibriz.ttf,
BoldFont      =calibrib.ttf,
ItalicFont    =calibrii.ttf]{calibri.ttf} % use original calibri font

\setmonofont[Path=/home/yourusername/.local/share/fonts/font-windows/courier-new/,
BoldItalicFont=courbi.ttf,
BoldFont      =courbd.ttf,
ItalicFont    =couri.ttf]{cour.ttf} % use original courir new font


% -----------------------------------------------------------------
% Debungging configuration
% -----------------------------------------------------------------
% \overfullrule=2cm % show overfull
% \usepackage{showframe} % show frame margin
% \renewcommand\ShowFrameLinethickness{0.25pt}
% \renewcommand*\ShowFrameColor{\color{red}}


% -----------------------------------------------------------------
% Bibliography style
% -----------------------------------------------------------------
\input{hangilastyle}
\addbibresource{daftar-pustaka.bib}

\renewcommand*{\finalnamedelim}{% % change and to &
  \ifnumgreater{\value{liststop}}{2}{\finalandcomma}{}%
  \addspace\&\space}

\newcommand{\cmmnt}[1]{\ignorespaces} % remove space in generated
% documents if we put comments in source

% -----------------------------------------------------------------
% Fill your data here
% -----------------------------------------------------------------
% title
\titlepkl{The Compendious Book on Calculation by Completion and Balancing}
% personal data
\fullname{M. Ibn Musa al-Khwarizmi}
\idnum{112233445566}
\approvaldate{11}
\degree{N/A}
\yearsubmit{2018}
\laboratory{Rekayasa Perangkat Lunak}
\pklplacetype{PT}
\pklstartdate{1 Juli}
\pklenddate{31 Agustus 2018}
\pklplace{Foo Baz Bar}
\dept{TEKNIK INFORMATIKA}
\faculty{FAKULTAS ILMU KOMPUTER}
\university{universitas brawijaya}
\city{malang}
% supervisor data
\firstsupervisor{Ibn ar-Razāz al-Jazarī, S.T., M.T., Ph.D}
\firstsupervisornip{1234 45679}
\secondsupervisor{Ibn Mūsā ibn Shākir , S.T., M.T., Ph.D}
\secondsupervisornip{5555 7777}


% -----------------------------------------------------------------
% And so it begins
% -----------------------------------------------------------------
\begin{document}

% -----------------------------------------------------------------
% Cover
% -----------------------------------------------------------------
\cover

% -----------------------------------------------------------------
% Approval Page
% -----------------------------------------------------------------
\approvalpage

% -----------------------------------------------------------------
% Approval Page Lapangan
% -----------------------------------------------------------------
% \approvalpagelapangan


% -----------------------------------------------------------------
% PERNYATAAN ORISINALITAS
% -----------------------------------------------------------------
{\orisinalitas

  Saya menyatakan dengan sebenar-benarnya bahwa sepanjang pengetahuan
  saya, di dalam laporan PKL ini tidak terdapat karya ilmiah yang
  pernah diajukan oleh orang lain dalam kegiatan akademik di suatu
  perguruan tinggi, dan tidak terdapat karya atau pendapat yang pernah
  ditulis atau diterbitkan oleh orang lain, kecuali yang secara
  tertulis disitasi dalam naskah ini dan disebutkan dalam daftar
  pustaka.

  Apabila ternyata didalam laporan PKL ini terbukti terdapat
  unsur-unsur plagiasi, saya bersedia PKL ini digugurkan, serta
  diproses sesuai dengan peraturan perundang-undangan yang berlaku (UU
  No. 20 Tahun 2003, Pasal 25 ayat 2 dan Pasal 70).
  \vspace{1.5cm}

  \noindent
  % \hspace*{8cm}Malang, 31 Agustus 2018 \vspace{1.5cm} \\
  \hspace*{8cm}Malang, 31 Agustus 2018  \\
  \hspace*{8cm}Ketua Kelompok,  \vspace{1.5cm} \\

  \hspace*{6.8cm}\underline{M. Ibn Musa al-Khwarizmi} \\
  \hspace*{8cm}NIM : 112233445566

}

% -----------------------------------------------------------------
% Disini awal masukan untuk Prakata
% -----------------------------------------------------------------
{\preface

  Puji syukur kehadirat Allah SWT yang telah melimpahkan
  rahmat, taufik dan hidayah-Nya sehingga laporan PKL yang berjudul
  “Rancang Bangun Aplikasi Jejaring Sosial Kampus Berbasis GPS Pada
  Smartphone Android” ini dapat terselesaikan.

  \begin{enumerate}
  \item{Bapak Ibn ar-Razāz al-Jazarī, S.T., M.T., Ph.D. selaku dosen pembimbing PKL
      yang telah dengan sabar membimbing dan mengarahkan penulis
      sehingga dapat menyelesaikan laporan ini.}
  \item{Bapak Ibn Mūsā ibn Shākir , S.T., M.T., Ph.D. selaku ketua pembimbing lapangan.}
  \item{Ayahanda dan Ibunda dan seluruh keluarga besar atas segala
      nasehat, kasih sayang, perhatian dan kesabarannya di dalam
      membesarkan dan mendidik penulis, serta yang senantiasa tiada
      henti-hentinya memberikan doa dan semangat demi terselesaikannya
      laporan ini.}
  \item{Seluruh civitas akademika Teknik Informatika Universitas
      Brawijaya yang telah banyak memberi bantuan dan dukungan selama
      penyelesaian laporan PKL ini.}
  \end{enumerate}

  Penulis menyadari bahwa dalam penyusunan laporan ini masih banyak kekurangan,
  sehingga saran dan kritik yang membangun sangat penulis harapkan. Akhir kata penulis
  berharap PKL ini dapat membawa manfaat bagi semua pihak yang menggunakannya.


  \vspace{0.8cm}


  \noindent
  \hspace*{8cm}Malang, 31 Agustus 2018 \\
  \hspace*{8cm}Ketua Kelompok, \vspace{1.5cm} \\

  \hspace*{6.8cm}M. Ibn Musa al-Khwarizmi \\
  \hspace*{8cm}Email: alkhwarizmi@student.ub.ac.id

}

% -----------------------------------------------------------------
% ABSTRAK INDONESIA
% -----------------------------------------------------------------
{\abstractind

  Pellentesque non quam in ipsum \emph{rutrum cursus} consequat mollis
  mi. Maecenas non sapien luctus magna mollis ultrices. Curabitur
  pulvinar, lorem a pharetra volutpat, orci lorem dictum nisi, vel
  rutrum felis elit eu risus. In hac habitasse platea
  dictumst. Quisque eget sollicitudin tellus, laoreet sodales
  purus. Sed lobortis ornare nisi non eleifend. Aenean viverra
  placerat est, id facilisis metus tincidunt eu. In velit ligula,
  ultrices vel eros eu, viverra consequat augue. Donec ac turpis
  justo.

  Maecenas posuere convallis ligula, ut ullamcorper orci pulvinar
  in. Proin libero nisl, dapibus at congue venenatis, lacinia quis
  diam. Nulla sed facilisis sapien. Fusce vitae enim auctor, convallis
  metus eu, laoreet augue. Suspendisse nec hendrerit tortor. Proin
  ultricies gravida lorem, eu lobortis turpis dictum nec. In hac
  habitasse platea dictumst. Quisque urna nunc, feugiat eget lorem
  vitae, eleifend lobortis lacus. Integer congue, arcu nec porttitor
  venenatis, ligula enim molestie neque, eu venenatis orci leo eu
  tortor. Etiam consectetur augue sed mi molestie suscipit. Donec est
  ligula, tempus eget egestas sed, ornare a erat.

  Ut a fringilla felis, ut \emph{rutrum libero}. Nulla tempus urna ex, sit
  amet consequat massa accumsan tincidunt. Etiam aliquet mauris sit
  amet tortor sagittis, sed vehicula dui tristique. In hac habitasse
  platea dictumst. Donec eget arcu blandit, accumsan ex a, pharetra
  dui. Aliquam pharetra dui libero, eget lacinia lacus aliquet
  quis. Aenean bibendum nunc condimentum volutpat gravida. Duis
  facilisis neque sit amet lectus venenatis.

  \emph{Kata kunci}: fringilla, \emph{venenatis}, \emph{volutpat gravida}


}

% -----------------------------------------------------------------
% ABSTRAK ENGLSIH
% -----------------------------------------------------------------
{\abstracteng

  \emph{Pellentesque non quam in ipsum rutrum cursus consequat mollis
    mi. Maecenas non sapien luctus magna mollis ultrices. Curabitur
    pulvinar, lorem a pharetra volutpat, orci lorem dictum nisi, vel
    rutrum felis elit eu risus. In hac habitasse platea
    dictumst. Quisque eget sollicitudin tellus, laoreet sodales
    purus. Sed lobortis ornare nisi non eleifend. Aenean viverra
    placerat est, id facilisis metus tincidunt eu. In velit ligula,
    ultrices vel eros eu, viverra consequat augue. Donec ac turpis
    justo.}

  \emph{Maecenas posuere convallis ligula, ut ullamcorper orci pulvinar
    in. Proin libero nisl, dapibus at congue venenatis, lacinia quis
    diam. Nulla sed facilisis sapien. Fusce vitae enim auctor, convallis
    metus eu, laoreet augue. Suspendisse nec hendrerit tortor. Proin
    ultricies gravida lorem, eu lobortis turpis dictum nec. In hac
    habitasse platea dictumst. Quisque urna nunc, feugiat eget lorem
    vitae, eleifend lobortis lacus. Integer congue, arcu nec porttitor
    venenatis, ligula enim molestie neque, eu venenatis orci leo eu
    tortor. Etiam consectetur augue sed mi molestie suscipit. Donec est
    ligula, tempus eget egestas sed, ornare a erat.}

  \emph{Ut a fringilla felis, ut rutrum libero. Nulla tempus urna ex, sit
    amet consequat massa accumsan tincidunt. Etiam aliquet mauris sit
    amet tortor sagittis, sed vehicula dui tristique. In hac habitasse
    platea dictumst. Donec eget arcu blandit, accumsan ex a, pharetra
    dui. Aliquam pharetra dui libero, eget lacinia lacus aliquet
    quis. Aenean bibendum nunc condimentum volutpat gravida. Duis
    facilisis neque sit amet lectus venenatis.}

  \emph{Keyword}: fringilla, \emph{venenatis}, \emph{volutpat gravida}
}

% -----------------------------------------------------------------
% TOCS
% -----------------------------------------------------------------
\tableofcontents
\addcontentsline{toc}{chapter}{DAFTAR ISI}
\listoftables
\addcontentsline{toc}{chapter}{DAFTAR TABEL}
\listoffigures
\addcontentsline{toc}{chapter}{DAFTAR GAMBAR}
% TODO \listofappendices
\addcontentsline{toc}{chapter}{DAFTAR LAMPIRAN}

% -----------------------------------------------------------------
% Chapter
% -----------------------------------------------------------------
% change numbering to arabic from bab1
\selectlanguage{bahasa}\clearpage\pagenumbering{arabic}\setcounter{page}{1}

\include{bab1}
%%%%%%%%%%%%%%%%%%%%%%%%%%%%%%%%%%%%%%%%%%%%%%%%%%%%%%%%%%%%%%%%%%%%%%%
% BAB 2
%%%%%%%%%%%%%%%%%%%%%%%%%%%%%%%%%%%%%%%%%%%%%%%%%%%%%%%%%%%%%%%%%%%%%%%

\mychapter{2}{BAB 2 GAMBARAN UMUM INSTANSI}

Curabitur elementum consequat elementum. Integer in vulputate
metus. Aliquam erat volutpat. Phasellus vitae convallis orci. Etiam ac
metus tristique, viverra metus a, mattis lectus. Aenean vel vestibulum
augue.

\section{Tinjauan Pustaka}

% comments
%% comments
Phasellus porttitor purus \textcite{warn} vitae finibus laoreet. Class
aptent taciti sociosqu ad litora torquent per conubia nostra, per
inceptos himenaeos. Lorem ipsum dolor sit amet, consectetur adipiscing
elit. Nunc tempus sed purus sit amet porta. Sed volutpat est ac arcu
porttitor, non pharetra tortor pharetra. Phasellus vitae dictum
elit. Curabitur elementum consequat elementum. Integer in vulputate
metus. Aliquam erat volutpat. Phasellus vitae convallis orci. Etiam ac
metus tristique, viverra metus a, mattis lectus. Aenean vel vestibulum
augue.

\section{Landasan Teori}

\subsection{\emph{Lorem}}

Curabitur elementum consequat elementum. Integer in vulputate
metus. Aliquam erat volutpat. Phasellus vitae convallis orci. Etiam ac
metus tristique, viverra metus a, mattis lectus. Aenean vel vestibulum
augue.

\subsection{\emph{Ipsum}}

Curabitur elementum consequat elementum. Integer in vulputate
metus. Aliquam erat volutpat. Phasellus vitae convallis orci. Etiam ac
metus tristique, viverra metus a, mattis lectus. Aenean vel vestibulum
augue.

\subsection{\emph{Dolor sit}}

Curabitur elementum consequat elementum. Integer in vulputate
metus. Aliquam erat volutpat. Phasellus vitae convallis orci. Etiam ac
metus tristique, viverra metus a, mattis lectus. Aenean vel vestibulum
augue.

\newpage

\begin{center}
\begin{minipage}{0.8\textwidth}
\begin{code}
\begin{ignasicblock}[title=test\_hallo,minted language=Python]
      def test_case_1():
          assert hallo("Budi") == "Hai Budi"

      def test_case_2():
          assert hallo("Ani") == "Nama Kosong"
\end{ignasicblock}
\captionof{listing}{\emph{Contoh script atomated testing}}
\label{ts:hallo}
\end{code}
\end{minipage}
\end{center}


\section{Teknologi Pengembangan Perangkat Lunak}

\subsection{\emph{Curabitur}}

Curabitur elementum consequat elementum. Integer in vulputate
metus. Aliquam erat volutpat. Phasellus vitae convallis orci. Etiam ac
metus tristique, viverra metus a, mattis lectus. Aenean vel vestibulum
augue.

\subsection{\emph{Elementum}}

Curabitur elementum consequat elementum. Integer in vulputate
metus. Aliquam erat volutpat. Phasellus vitae convallis orci. Etiam ac
metus tristique, viverra metus a, mattis lectus. Aenean vel vestibulum
augue.

\include{bab3}
% \include{bab4}
% \include{bab5}
% \include{bab6}

% -----------------------------------------------------------------
% Bibliography
% -----------------------------------------------------------------
% -- \addcontentsline{toc}{chapter}{DAFTAR PUSTAKA}
\emergencystretch=1em
\printbibliography[title={DAFTAR PUSTAKA}]

% \include{lampiran}

\end{document}

%%% Local Variables:
%%% mode: latex
%%% TeX-master: t
%%% End:
